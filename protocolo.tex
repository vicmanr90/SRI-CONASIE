\documentclass[12pt]{article}
\usepackage[a4paper,margin=2.5cm]{geometry}
\usepackage{lmodern}
\usepackage{titlesec}
\usepackage{graphicx}
\sloppy

\title{Protocolo SRI}
\author{Andrés Ballesta y Shiu Avila}
\date{May 2025}

\begin{document}

\maketitle

\section{Protocolo}
En el presente trabajo de investigaci\'on acerca de las inversiones socialmente responsables (SRI) y el estado del arte de herramientas de evaluación de portafolios sostenibles, se realiz\'o una búsqueda en las bases de datos Scielo y Scopus, empleando la expresi\'on \textbf{("financial" AND ("investment" OR "investments"))
AND ("SRI" OR "responsible" OR "ethical" OR "sustainable")
AND ("indicators" OR "measurement" OR "metrics" OR "indexes" OR "frameworks")}. Sin embargo, esta expresi\'on de b\'usqueda no focalizaba con coherencia el punto principal de la investigaci\'on, Esta situaci\'on se manifest\'o al aplicar la expresi\'on de b\'usqueda en la base de datos Scopus, obteni\'endose un total de 3861 art\'iculos, de los cuales una proporci\'on significativa result\'o no ser pertinente para los fines espec\'ificos del estudio.\\
\\
Posteriormente se agregaron t\'erminos como "performance indicators", "quantitative indicators", "impact metrics", "sustainability performance metrics", entre otros que permitieron filtrar estudios centrados en an\'alisis cuantitativos y emp\'iıricos, atraer art\'iıculos relacionados con la construcc\'ion o c\'alculo de \'indices sostenibles y añadir precisi\'on metodol\'ogica a la investigaci\'on, quedando de la siguiente forma \textbf{("financial" AND ("investment" OR "investments"))
AND ( "SRI" OR "responsible" OR "ethical" OR "sustainable" ) AND ("performance indicators" OR "quantitative indicators" OR "impact metrics" OR "sustainability performance metrics" OR "measurement framework" OR "assessment framework" OR "index methodology" OR "scoring system" OR "rating system" OR "benchmarking" OR "evaluation model" OR "comparative analysis" OR "criteria" OR "responsible investment metrics" OR "SRI metrics" OR "SRI framework" OR "SRI assessment")}. Esta nueva expresi\'on redujo los resultados a 1242 art\'iculos.\\
\\
Luego, con apoyo en la inteligencia artificial, se delimit\'o a\'un m\'as el alcance de los resultados, quedando estos m\'as focalizados al objeto de investigaci\'on. la expresi\'on de b\'usqueda refinada por la IA qued\'o de la siguiente manera \textbf{("financial" AND ("investment" OR "investments")) AND ("SRI" OR "socially responsible investing" OR "responsible investment" OR "ethical investment" OR "sustainable investment" OR "responsible finance") AND ("performance indicators" OR "quantitative indicators" OR "impact metrics" OR "sustainability metrics" OR "measurement framework" OR "assessment framework" OR "index methodology" OR "scoring system" OR "rating system" OR "benchmarking" OR "evaluation model" OR "comparative analysis" OR "criteria" OR "responsible investment metrics" OR "SRI metrics" OR "SRI framework" OR "SRI assessment")}, donde se reemplazaron e incorporaron t\'erminos y expresi\'ones completas del enfoque SRI que describen estrategias de inversi\'on basadas en valores \'eticos, lo que redujo la cifra de resultados a 314 documentos.\\
\\
Seguidamente se combin\'o la expresi\'on de b\'usqueda incial con la refinada por la IA, obteniendo como resultado lo siguiente \textbf{( ( "financial" AND ( "investment" OR "investments" ) ) OR ( "portfolio" AND ( "management" OR "planning" ) ) ) AND ( "SRI" OR "socially responsible investing" OR "responsible investment" OR "ethical investment" OR "sustainable investment" OR "responsible finance" ) AND ( "performance indicators" OR "quantitative indicators" OR "impact metrics" OR "sustainability performance metrics" OR "measurement framework" OR "assessment framework" OR "index methodology" OR "scoring system" OR "rating system" OR "benchmarking" OR "evaluation model" OR "comparative analysis" OR "criteria" OR "responsible investment metrics" OR "SRI metrics" OR "SRI framework" OR "SRI assessment" )}, considerando solo art\'iculos y revisiones en ingl\'es, español y portugu\'es. Se excluyeron \'areas como qu\'imica, farmacolog\'ia, ciencias de la tierra, agr\'icolas y planetarias, inmunolog\'ia, f\'isica y astronom\'ia. Reduciendo as\'i la cantidad de resultados a 227 documentos.\\
\\
Una vez obtenidos los resultados depurados por medio de los criterios de b\'usqueda, se procedi\'o a aplicar un primer filtro basado en los t\'itulos de los art\'iculos, descartando aquellos que no presentaban una relaci\'on directa o relevante con el tema en cuesti\'on. Como resultado de esta etapa inicial, se conservaron 115 art\'iculos. A continuaci\'on, se llev\'o a cabo una segunda revisi\'on centrada en los res\'umenes (\textit{abstracts}) de estos documentos, seleccionando \'unicamente aquellos que hicieran referencia a \'indices, metodolog\'ias de evaluaci\'on, m\'etricas, an\'alisis de \textit{ratings}, modelos y dem\'as formas de medici\'on del SRI. Esta segunda revisi\'on arroj\'o un total de 60 art\'iculos pertinentes para el desarrollo del estudio.\\
\\
Con el fin de orientar el desarrollo de esta revisi\'on y de analizar las metodolog\'ias, herramientas y enfoques utilizados en la evaluaci\'on de portafolios sostenibles en el contexto de las inversiones socialmente responsables (SRI), se plantearon las siguientes preguntas de investigaci\'on:\\

\begin{itemize}
    \item ¿Cu\'ales son las principales metodolog\'ias utilizadas en la evaluaci\'on de portafolios sostenibles y c\'omo difieren en t\'erminos de criterios y m\'etricas aplicadas?
    \item ¿Cu\'ales son las principales limitaciones y cr\'iticas identificadas en la literatura respecto al uso de herramientas de evaluaci\'on de sostenibilidad en el SRI?
    \item ¿Qu\'e factores o variables concretas se han estudiado con mayor frecuencia en el uso de herramientas de evaluación de sostenibilidad y c\'omo han sido medidas?
    \item ¿Qu\'e relaciones o interdependencias se identifican entre los factores m\'as representativos dentro de las herramientas de evaluaci\'on de sostenibilidad?
    \item ¿Qu\'e modelos te\'oricos o frameworks se utilizan con mayor frecuencia en las investigaciones sobre la evaluaci\'on de portafolios sostenibles?
    \item ¿C\'omo se han clasificado y comparado los factores evaluativos en las diferentes metodolog\'ias de an\'alisis de sostenibilidad financiera a lo largo del tiempo? 
\end{itemize}

\section{Metodolog\'ia}

\end{document}