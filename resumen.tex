\documentclass[12pt,a4paper]{article}

% Encoding and language
\usepackage[utf8]{inputenc}
\usepackage[T1]{fontenc}
\usepackage[spanish]{babel}

% Page geometry
\usepackage[a4paper,margin=2.5cm]{geometry}

% Math and symbols
\usepackage{amsmath,amssymb}

% Graphics
\usepackage{graphicx}

% Hyperlinks
\usepackage{hyperref}

% Other useful packages
\usepackage{enumitem}
\usepackage{booktabs}

% Title, author, date
\title{Resumen}
\author{Shiu Avila, Andrés Ballesta, Victor Noble}
\date{\today}

\begin{document}

\maketitle

\section{Planteamiento del problema}
En un contexto global marcado por la escasez de recursos y la creciente presión social y regulatoria, las empresas enfrentan un escrutinio más riguroso sobre sus impactos ambientales y sociales. Invertir en organizaciones que ignoran la sostenibilidad o muestran deficiencias en gobernanza no solo limita su competitividad, sino que también implica riesgos legales, reputacionales y financieros (Chytis, Eriotis, \& Mitroulia, 2024). Esta situación ha favorecido el auge de la inversión socialmente responsable (SRI), que integra criterios ambientales, sociales y de gobernanza (ESG) como parte de la estrategia de asignación de capital. Dichos enfoques buscan no solo la rentabilidad, sino también la generación de impactos positivos, fortaleciendo la resiliencia y el valor a largo plazo de las organizaciones (Rubab, Ali, \& Khan, 2025; González-Pozo, Jiménez, \& Morales, 2024).\\
\\
Sin embargo, la evaluación de portafolios sostenibles enfrenta retos importantes. Por un lado, los beneficios derivados de la sostenibilidad suelen manifestarse en el largo plazo, mientras que los mercados financieros tienden a priorizar resultados inmediatos. Por otro lado, los sistemas de calificación ESG presentan alta heterogeneidad y falta de estandarización, lo cual genera discrepancias entre las agencias evaluadoras. Por ejemplo, organismos como Refinitiv, Bloomberg, MSCI o Sustainalytics aplican criterios y metodologías distintas, lo que ocasiona resultados divergentes incluso para una misma empresa (Cesarone, Colucci, Colucci, \& Tardella, 2024). A esto se suman problemas como la opacidad metodológica, la subjetividad en la ponderación de criterios y el riesgo de \textit{greenwashing}, que dificultan la comparabilidad entre sectores y regiones (Estevez-Mendoza \& Infante, 2024; Rubab et al., 2025).\\  
\\
Estas inconsistencias tienen efectos directos en la selección de portafolios y en la construcción de fronteras eficientes de riesgo-retorno, pues las diferencias en las calificaciones pueden alterar las decisiones de inversión (Cesarone et al., 2024). Aun así, el interés en la inversión sostenible sigue en expansión, apoyado por regulaciones como la Taxonomía Verde y el Pacto Verde Europeo, así como por el creciente número de signatarios de los Principios de Inversión Responsable (PRI) (Chytis et al., 2024; González-Pozo et al., 2024).\\  
\\
De este modo, se hace necesario analizar críticamente las metodologías actuales, identificar las variables más estudiadas y evaluar el impacto que tienen las divergencias entre calificaciones ESG en la toma de decisiones financieras. Asimismo, cobra relevancia la exploración de modelos alternativos y enfoques innovadores, como el método Best-Worst extendido o las medidas de proximidad ordinal, que buscan mejorar la coherencia, transparencia y robustez de las evaluaciones ESG (González-Pozo et al., 2024; Cesarone et al., 2024).
\section{Justificación}

\section{Objetivo General}
\section{Objetivos Específicos}


\section{Marco Teórico}
\section{Metodología}

Wenas.

\section{Resultados parciales}
\section{Referencias}

\end{document}